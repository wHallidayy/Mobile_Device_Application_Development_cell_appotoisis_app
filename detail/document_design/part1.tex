\section{ความต้องการทั้งหมด + สิ่งที่จะทำในรายวิชานี้}

\subsection{ภาพรวมของระบบ}

\subsection{วิเคราะห์ความต้องการเป็นฟังก์ชัน (Functional Requirements)}

\subsubsection{Authentication Module (ระบบยืนยันตัวตน)}
\begin{itemize}
    \item ระบบลงทะเบียนผู้ใช้ใหม่ (Register) พร้อมข้อมูลพื้นฐาน
    \item ระบบเข้าสู่ระบบ (Login) ด้วย Username และ Password
    \item ระบบออกจากระบบ (Logout)
    \item แยก Account ของผู้ใช้แต่ละคนโดยเด็ดขาด ไม่สามารถแชร์ข้อมูลระหว่างผู้ใช้ได้
\end{itemize}

\subsubsection{Folder Management Module (ระบบจัดการโฟลเดอร์)}
\begin{itemize}
    \item สร้างโฟลเดอร์ใหม่ (Create Folder)
    \item แก้ไชื่อโฟลเดอร์ (Rename Folder)
    \item ลบโฟลเดอร์ (Delete Folder)
    \item แสดงรายการโฟลเดอร์ทั้งหมดของผู้ใช้
    \item \textbf{กฎพิเศษ:} โฟลเดอร์มีโครงสร้างแบบ Flat (ไม่สามารถซ้อนโฟลเดอร์ได้)
\end{itemize}

\subsubsection{Image Management Module (ระบบจัดการภาพ)}
\begin{itemize}
    \item นำเข้าภาพจาก Gallery ของอุปกรณ์
    \item ดูรายละเอียด Metadata ของภาพ (ขนาดไฟล์, ขนาดภาพ)
    \item ลบภาพออกจากระบบ
    \item \textbf{กฎพิเศษ:} ภาพทุกภาพต้องถูกจัดเก็บในโฟลเดอร์ใดโฟลเดอร์หนึ่งเสมอ (ห้ามมีภาพที่ไม่อยู่ในโฟลเดอร์)
\end{itemize}

\subsubsection{AI Analysis Integration Module (ระบบวิเคราะห์ภาพด้วย AI)}
\begin{itemize}
    \item ส่งภาพไปยัง Deep Learning Classification Model ผ่าน API
    \item รับผลการวิเคราะห์และแสดงผล โดยผลลัพธ์จะจำแนกเป็น 3 คลาส:
    \begin{itemize}
        \item \textbf{Viable:} เซลล์มีชีวิตปกติ
        \item \textbf{Apoptosis:} เซลล์กำลังตาย
        \item \textbf{Other:} เซลล์หรือวัตถุที่ Model ตรวจพบแต่ไม่อยู่ในสองกลุ่มแรก
    \end{itemize}
    \item บันทึกผลการวิเคราะห์ลงฐานข้อมูลพร้อมกับภาพ
    \item แสดงประวัติการวิเคราะห์ของแต่ละภาพ
\end{itemize}

\subsection{Non-Functional Requirements (ความต้องการด้านคุณภาพ)}
\begin{itemize}
    \item \textbf{ความปลอดภัย (Security):} ข้อมูลแต่ละ Account แยกกันโดยสิ้นเชิง
    \item \textbf{ความเป็นส่วนตัว (Privacy):} ผู้ใช้ไม่สามารถแชร์ผลการวิเคราะห์จาก AI ให้ผู้อื่นได้
    \item \textbf{ประสิทธิภาพ (Performance):} การประมวลผล AI ต้องไม่ทำให้แอปค้าง
    \item \textbf{ความใช้งานง่าย (Usability):} UI/UX เหมาะสมกับผู้ใช้ที่เป็นนักวิจัยและนักศึกษา
\end{itemize}