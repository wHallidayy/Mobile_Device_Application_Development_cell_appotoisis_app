\section{ออกแบบแอป}

\subsection{User Case Diagram}

\begin{figure}[h]
    \centering
    \includegraphics[width=0.5\textwidth]{assets/usecase_diagram.pdf}
    \caption{UseCase Diagram ของระบบจัดการภาพเซลล์}
    \label{fig:usecase_diagram}
\end{figure}

% ============================ mermaid ============================
% flowchart LR

% subgraph Auth["Authentication"]

% UC1(("Register"))

% UC2(("Login"))

% UC3(("Logout"))

% end

% subgraph Folder["Folder Management"]

% UC4(("Create Folder"))

% UC5(("Rename Folder"))

% UC6(("Delete Folder"))

% UC7(("View Folder List"))

% end

% subgraph Image["Image Management"]

% UC8(("Import image"))

% UC11(("Delete Image"))

% end

% subgraph Analysis["Analysis"]

% UC12(("Analyze Image"))

% UC13(("View Analysis History"))

% UC14(("Save Analysis Result"))

% end

% subgraph System["Cell Analysis web Application"]

% direction TB

% Auth

% Folder

% Image

% Analysis

% end

% User["👤 User<br>(Researcher/Student/Lecturer)"] --> UC1 & UC2 & UC3 & UC4 & UC5 & UC6 & UC7 & UC8 & UC11 & UC12 & UC13

% UC12 -. include .-> UC14

\clearpage

\subsection{Activity Diagram}

\begin{figure}[h]
    \centering
    \includegraphics[width=0.4\textwidth]{assets/activity_diagram.pdf}
    \caption{Activity Diagram ของระบบจัดการภาพเซลล์}
    \label{fig:activity_diagram}
\end{figure}


% flowchart TD
%     %% Define Swimlanes using Subgraphs
%     subgraph UserApp [User]
%         Start((Start)) --> Login[เข้าสู่ระบบ Users]
%         Login --> Folder[เลือก/สร้างโฟลเดอร์]
%         Folder --> SelectImg[เลือกรูปภาพ]
%         SelectImg --> Request[กดปุ่มวิเคราะห์ Request]
%     end

%     subgraph Backend [Backend API]
%         Request --> Validate{ไฟล์ถูกต้อง?}
%         Validate -- No --> Error[แจ้งเตือน Error]
%         Error --> Stop((End))
        
%         Validate -- Yes --> SaveImg[บันทึกรูปภาพ INSERT Images]
%         SaveImg --> CreateJob[สร้าง Job 'pending' INSERT Jobs]
%         CreateJob --> TriggerModal[ส่งงานเข้า Modal Queue]
%         TriggerModal --> Return202[ตอบกลับทันที 202 Accepted]
%     end

%     subgraph Polling [App Polling Process]
%         Return202 --> ShowProcess[แสดงสถานะ 'กำลังประมวลผล']
%         ShowProcess --> CheckStatus{ตรวจสอบสถานะงาน?}
%         CheckStatus -- Pending --> Wait[รอสักครู่...]
%         Wait --> CheckStatus
        
%         CheckStatus -- Completed --> GetResult[ดึงผลลัพธ์ SELECT Analysis_Results]
%         GetResult --> ShowGraph[แสดงกราฟ/ผลการนับ]
%         ShowGraph --> Finish((End))
%     end

%     subgraph AIWorker [AI Worker Modal]
%         TriggerModal -.-> AIStart[AI รับงาน & เริ่มประมวลผล]
%         AIStart --> Classify[จำแนกประเภทเซลล์]
        
%         subgraph DB_Action [Database Operations]
%             Classify --> SaveResult[บันทึกผล INSERT Analysis_Results]
%             SaveResult --> UpdateJob[อัปเดต Job 'completed']
%         end
%     end
    
%     %% Link Asynchronous update back to data check implicitly
%     UpdateJob -.-> CheckStatus  

\clearpage

\subsection{Sequence Diagram}

\begin{figure}[h]
    \centering
    \includegraphics[width=0.9\textwidth]{assets/sequence_diagram.pdf}
    \caption{Sequence Diagram ของระบบจัดการภาพเซลล์}
    \label{fig:sequence_diagram}
\end{figure}

% sequenceDiagram
%     autonumber
%     actor User
%     participant App as frontend
%     participant API as Backend API
%     participant DB as Database
%     participant AI as AI Model Service

%     Note over User, App: ผู้ใช้เลือกรูปภาพและกดปุ่ม "Analyze"

%     User->>App: Request Analysis (image_id)
%     activate App
    
%     App->>API: POST /analyze {image_id, model_version}
%     activate API

%     %% ขั้นตอนดึงข้อมูลภาพ
%     API->>DB: Query Image Path (SELECT * FROM Images WHERE id=...)
%     activate DB
%     DB-->>API: Return file_path & metadata
%     deactivate DB

%     %% ขั้นตอนส่งไป AI
%     Note over API, AI: ส่งภาพไปประมวลผลตาม Functional Req 1.2.4 
%     API->>AI: Send Image Data
%     activate AI
%     AI-->>API: Return Result {Viable, Apoptosis, Other, Confidence}
%     deactivate AI

%     %% ขั้นตอนบันทึกผลลงฐานข้อมูล
%     Note right of API: บันทึกผลลงตาราง Jobs และ Analysis_Results 
%     API->>DB: INSERT INTO Jobs (status='completed', ...)
%     API->>DB: INSERT INTO Analysis_Results (counts, raw_data)
    
%     %% ส่งผลกลับ Client
%     API-->>App: Return Analysis JSON
%     deactivate API

%     App-->>User: Display Results (Graph/Text)
%     deactivate App