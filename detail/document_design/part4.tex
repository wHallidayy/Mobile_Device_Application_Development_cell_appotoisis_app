\section{API design}

\subsection{API Overview}

ระบบใช้สถาปัตยกรรม RESTful API โดยมี Base URL คือ \texttt{/api/v1} 
การยืนยันตัวตนใช้ JWT (JSON Web Token) ผ่าน Header: \texttt{Authorization: Bearer <token>}

\subsection{Response Format มาตรฐาน}

ทุก Response จะอยู่ในรูปแบบ JSON โดยมีโครงสร้างดังนี้:

\begin{verbatim}
{
  "success": true/false,
  "data": { ... },       // กรณีสำเร็จ
  "error": {             // กรณีล้มเหลว
    "code": "ERROR_CODE",
    "message": "รายละเอียดข้อผิดพลาด"
  }
}
\end{verbatim}

\subsection{HTTP Status Codes}

\begin{table}[h!]
    \centering
    \begin{tabular}{|c|l|}
        \hline
        \textbf{Status Code} & \textbf{ความหมาย} \\
        \hline
        200 OK & ดำเนินการสำเร็จ \\
        \hline
        201 Created & สร้างข้อมูลสำเร็จ \\
        \hline
        202 Accepted & รับคำขอแล้ว (สำหรับ Async Job) \\
        \hline
        400 Bad Request & ข้อมูลไม่ถูกต้อง \\
        \hline
        401 Unauthorized & ไม่ได้เข้าสู่ระบบ \\
        \hline
        403 Forbidden & ไม่มีสิทธิ์เข้าถึง \\
        \hline
        404 Not Found & ไม่พบข้อมูล \\
        \hline
        500 Internal Server Error & เกิดข้อผิดพลาดภายในระบบ \\
        \hline
    \end{tabular}
    \caption{HTTP Status Codes ที่ใช้ใน API}
    \label{tab:http_status}
\end{table}

\clearpage

%% ==================== AUTHENTICATION ====================
\subsection{Authentication API}

\subsubsection{ลงทะเบียนผู้ใช้ใหม่ (Register)}

\begin{table}[h!]
    \centering
    \begin{tabular}{|l|l|}
        \hline
        \textbf{Endpoint} & \texttt{POST /api/v1/auth/register} \\
        \hline
        \textbf{Authentication} & ไม่ต้อง \\
        \hline
    \end{tabular}
\end{table}

\textbf{Request Body:}
\begin{verbatim}
{
  "username": "string (required, unique)",
  "password": "string (required, min 8 chars)"
}
\end{verbatim}

\textbf{Response (201 Created):}
\begin{verbatim}
{
  "success": true,
  "data": {
    "user_id": "uuid",
    "username": "string",
    "created_at": "datetime"
  }
}
\end{verbatim}

\subsubsection{เข้าสู่ระบบ (Login)}

\begin{table}[h!]
    \centering
    \begin{tabular}{|l|l|}
        \hline
        \textbf{Endpoint} & \texttt{POST /api/v1/auth/login} \\
        \hline
        \textbf{Authentication} & ไม่ต้อง \\
        \hline
    \end{tabular}
\end{table}

\textbf{Request Body:}
\begin{verbatim}
{
  "username": "string",
  "password": "string"
}
\end{verbatim}

\textbf{Response (200 OK):}
\begin{verbatim}
{
  "success": true,
  "data": {
    "access_token": "jwt_token",
    "refresh_token": "jwt_token",
    "expires_in": 3600,
    "user": {
      "user_id": "uuid",
      "username": "string"
    }
  }
}
\end{verbatim}

\subsubsection{ออกจากระบบ (Logout)}

\begin{table}[h!]
    \centering
    \begin{tabular}{|l|l|}
        \hline
        \textbf{Endpoint} & \texttt{POST /api/v1/auth/logout} \\
        \hline
        \textbf{Authentication} & ต้อง (Bearer Token) \\
        \hline
    \end{tabular}
\end{table}

\textbf{Response (200 OK):}
\begin{verbatim}
{
  "success": true,
  "data": {
    "message": "Logged out successfully"
  }
}
\end{verbatim}

\clearpage

%% ==================== FOLDER MANAGEMENT ====================
\subsection{Folder Management API}

\subsubsection{แสดงรายการโฟลเดอร์ทั้งหมด}

\begin{table}[h!]
    \centering
    \begin{tabular}{|l|l|}
        \hline
        \textbf{Endpoint} & \texttt{GET /api/v1/folders} \\
        \hline
        \textbf{Authentication} & ต้อง (Bearer Token) \\
        \hline
    \end{tabular}
\end{table}

\textbf{Response (200 OK):}
\begin{verbatim}
{
  "success": true,
  "data": {
    "folders": [
      {
        "folder_id": 1,
        "folder_name": "Experiment 2024-01",
        "image_count": 25,
        "created_at": "2024-01-15T10:30:00Z"
      }
    ],
    "total": 10
  }
}
\end{verbatim}

\clearpage

\subsubsection{สร้างโฟลเดอร์ใหม่ (Create Folder)}

\begin{table}[h!]
    \centering
    \begin{tabular}{|l|l|}
        \hline
        \textbf{Endpoint} & \texttt{POST /api/v1/folders} \\
        \hline
        \textbf{Authentication} & ต้อง (Bearer Token) \\
        \hline
    \end{tabular}
\end{table}

\textbf{Request Body:}
\begin{verbatim}
{
  "folder_name": "string (required)"
}
\end{verbatim}

\textbf{Response (201 Created):}
\begin{verbatim}
{
  "success": true,
  "data": {
    "folder_id": 1,
    "folder_name": "string",
    "created_at": "datetime"
  }
}
\end{verbatim}

\subsubsection{แก้ไขชื่อโฟลเดอร์ (Rename Folder)}

\begin{table}[h!]
    \centering
    \begin{tabular}{|l|l|}
        \hline
        \textbf{Endpoint} & \texttt{PATCH /api/v1/folders/\{folder\_id\}} \\
        \hline
        \textbf{Authentication} & ต้อง (Bearer Token) \\
        \hline
    \end{tabular}
\end{table}

\textbf{Request Body:}
\begin{verbatim}
{
  "folder_name": "string (new name)"
}
\end{verbatim}

\textbf{Response (200 OK):}
\begin{verbatim}
{
  "success": true,
  "data": {
    "folder_id": 1,
    "folder_name": "updated_name",
    "created_at": "datetime"
  }
}
\end{verbatim}

\clearpage

\subsubsection{ลบโฟลเดอร์ (Delete Folder)}

\begin{table}[h!]
    \centering
    \begin{tabular}{|l|l|}
        \hline
        \textbf{Endpoint} & \texttt{DELETE /api/v1/folders/\{folder\_id\}} \\
        \hline
        \textbf{Authentication} & ต้อง (Bearer Token) \\
        \hline
    \end{tabular}
\end{table}

\textbf{หมายเหตุ:} การลบโฟลเดอร์จะลบรูปภาพและผลการวิเคราะห์ทั้งหมดในโฟลเดอร์นั้นด้วย (Cascade Delete)

\textbf{Response (200 OK):}
\begin{verbatim}
{
  "success": true,
  "data": {
    "message": "Folder deleted successfully",
    "deleted_images_count": 25
  }
}
\end{verbatim}

\clearpage

%% ==================== IMAGE MANAGEMENT ====================
\subsection{Image Management API}

\subsubsection{แสดงรายการรูปภาพในโฟลเดอร์}

\begin{table}[h!]
    \centering
    \begin{tabular}{|l|l|}
        \hline
        \textbf{Endpoint} & \texttt{GET /api/v1/folders/\{folder\_id\}/images} \\
        \hline
        \textbf{Authentication} & ต้อง (Bearer Token) \\
        \hline
        \textbf{Query Params} & \texttt{page=1\&limit=20} (optional) \\
        \hline
    \end{tabular}
\end{table}

\textbf{Response (200 OK):}
\begin{verbatim}
{
  "success": true,
  "data": {
    "images": [
      {
        "image_id": 1,
        "folder_id": 1,
        "original_filename": "cell_001.jpg",
        "file_size": 1024000,
        "mime_type": "image/jpeg",
        "has_analysis": true,
        "metadata": {
          "width": 1920,
          "height": 1080
        },
        "uploaded_at": "datetime"
      }
    ],
    "pagination": {
      "page": 1,
      "limit": 20,
      "total": 100,
      "total_pages": 5
    }
  }
}
\end{verbatim}

\clearpage

\subsubsection{อัปโหลดรูปภาพ (Import Image)}

\begin{table}[h!]
    \centering
    \begin{tabular}{|l|l|}
        \hline
        \textbf{Endpoint} & \texttt{POST /api/v1/folders/\{folder\_id\}/images} \\
        \hline
        \textbf{Authentication} & ต้อง (Bearer Token) \\
        \hline
        \textbf{Content-Type} & \texttt{multipart/form-data} \\
        \hline
    \end{tabular}
\end{table}

\textbf{Request Body (multipart/form-data):}
\begin{verbatim}
file: <binary image data>
\end{verbatim}

\textbf{Response (201 Created):}
\begin{verbatim}
{
  "success": true,
  "data": {
    "image_id": 1,
    "folder_id": 1,
    "original_filename": "cell_001.jpg",
    "file_size": 1024000,
    "mime_type": "image/jpeg",
    "has_analysis": false,
    "metadata": {
      "width": 1920,
      "height": 1080
    },
    "uploaded_at": "datetime"
  }
}
\end{verbatim}

\subsubsection{ดูรายละเอียดรูปภาพ}

\begin{table}[h!]
    \centering
    \begin{tabular}{|l|l|}
        \hline
        \textbf{Endpoint} & \texttt{GET /api/v1/images/\{image\_id\}} \\
        \hline
        \textbf{Authentication} & ต้อง (Bearer Token) \\
        \hline
    \end{tabular}
\end{table}

\textbf{Response (200 OK):}
\begin{verbatim}
{
  "success": true,
  "data": {
    "image_id": 1,
    "folder_id": 1,
    "original_filename": "cell_001.jpg",
    "file_url": "/api/v1/images/1/file",
    "file_size": 1024000,
    "mime_type": "image/jpeg",
    "metadata": {
      "width": 1920,
      "height": 1080,
      "captured_at": "datetime"
    },
    "analysis_history": [
      {
        "job_id": 1,
        "status": "completed",
        "ai_model_version": "v1.2.0",
        "finished_at": "datetime"
      }
    ],
    "uploaded_at": "datetime"
  }
}
\end{verbatim}

\subsubsection{ลบรูปภาพ (Delete Image)}

\begin{table}[h!]
    \centering
    \begin{tabular}{|l|l|}
        \hline
        \textbf{Endpoint} & \texttt{DELETE /api/v1/images/\{image\_id\}} \\
        \hline
        \textbf{Authentication} & ต้อง (Bearer Token) \\
        \hline
    \end{tabular}
\end{table}

\textbf{หมายเหตุ:} การลบรูปภาพจะลบ Jobs และ Analysis\_Results ที่เกี่ยวข้องด้วย

\textbf{Response (200 OK):}
\begin{verbatim}
{
  "success": true,
  "data": {
    "message": "Image deleted successfully"
  }
}
\end{verbatim}

\clearpage

\subsubsection{ขอ Presigned URL เพื่ออัปโหลด (Request Upload)}

\begin{table}[h!]
    \centering
    \begin{tabular}{|l|l|}
        \hline
        \textbf{Endpoint} & \texttt{POST /api/v1/folders/\{folder\_id\}/images/request-upload} \\
        \hline
        \textbf{Authentication} & ต้อง (Bearer Token) \\
        \hline
    \end{tabular}
\end{table}

\textbf{Request Body:}
\begin{verbatim}
{
  "filename": "image.jpg",
  "content_type": "image/jpeg",
  "file_size": 1024000
}
\end{verbatim}

\textbf{Response (200 OK):}
\begin{verbatim}
{
  "success": true,
  "data": {
    "upload_token": "unique_token_string",
    "presigned_url": "https://s3.endpoint/bucket/key?signature...",
    "expires_at": "datetime"
  }
}
\end{verbatim}

\subsubsection{ยืนยันการอัปโหลด (Confirm Upload)}

\begin{table}[h!]
    \centering
    \begin{tabular}{|l|l|}
        \hline
        \textbf{Endpoint} & \texttt{POST /api/v1/folders/\{folder\_id\}/images/confirm-upload} \\
        \hline
        \textbf{Authentication} & ต้อง (Bearer Token) \\
        \hline
    \end{tabular}
\end{table}

\textbf{Request Body:}
\begin{verbatim}
{
  "upload_token": "unique_token_string",
  "filename": "image.jpg",
  "content_type": "image/jpeg",
  "file_size": 1024000
}
\end{verbatim}

\textbf{Response (201 Created):}
\begin{verbatim}
{
  "success": true,
  "data": {
    "image_id": 1,
    "folder_id": 1,
    "original_filename": "image.jpg",
    "file_size": 1024000,
    "mime_type": "image/jpeg",
    "has_analysis": false,
    "uploaded_at": "datetime"
  }
}
\end{verbatim}

\clearpage

\subsubsection{ขอ Presigned URL เพื่อดาวน์โหลด (Get Download URL)}

\begin{table}[h!]
    \centering
    \begin{tabular}{|l|l|}
        \hline
        \textbf{Endpoint} & \texttt{GET /api/v1/images/\{image\_id\}/download-url} \\
        \hline
        \textbf{Authentication} & ต้อง (Bearer Token) \\
        \hline
    \end{tabular}
\end{table}

\textbf{Response (200 OK):}
\begin{verbatim}
{
  "success": true,
  "data": {
    "url": "https://s3.endpoint/bucket/key?signature...",
    "expires_at": "datetime"
  }
}
\end{verbatim}

\subsubsection{รายการรูปภาพแบบ V2 (Cursor Pagination)}

\begin{table}[h!]
    \centering
    \begin{tabular}{|l|l|}
        \hline
        \textbf{Endpoint} & \texttt{GET /api/v2/folders/\{folder\_id\}/images} \\
        \hline
        \textbf{Authentication} & ต้อง (Bearer Token) \\
        \hline
        \textbf{Query Params} & \texttt{cursor=timestamp\&limit=20} \\
        \hline
    \end{tabular}
\end{table}

\textbf{Response (200 OK):}
\begin{verbatim}
{
  "success": true,
  "data": {
    "images": [ ... ],
    "pagination": {
      "has_next": true,
      "next_cursor": "2024-01-24T10:00:00Z",
      "count": 20
    }
  }
}
\end{verbatim}

\clearpage

%% ==================== AI ANALYSIS ====================
\subsection{AI Analysis API}

\subsubsection{ส่งภาพเข้าวิเคราะห์ (Analyze Image)}

\begin{table}[h!]
    \centering
    \begin{tabular}{|l|l|}
        \hline
        \textbf{Endpoint} & \texttt{POST /api/v1/images/\{image\_id\}/analyze} \\
        \hline
        \textbf{Authentication} & ต้อง (Bearer Token) \\
        \hline
    \end{tabular}
\end{table}

\textbf{หมายเหตุ:} API นี้เป็นแบบ Asynchronous โดยจะตอบกลับทันทีด้วย 202 Accepted และ job\_id สำหรับติดตามสถานะ

\textbf{Request Body (optional):}
\begin{verbatim}
{
  "model_version": "v1.2.0"  // default: latest
}
\end{verbatim}

\textbf{Response (202 Accepted):}
\begin{verbatim}
{
  "success": true,
  "data": {
    "job_id": 1,
    "image_id": 1,
    "status": "pending",
    "ai_model_version": "v1.2.0",
    "status_url": "/api/v1/jobs/1",
    "created_at": "datetime"
  }
}
\end{verbatim}

\clearpage

\subsubsection{ตรวจสอบสถานะ Job (Check Job Status)}

\begin{table}[h!]
    \centering
    \begin{tabular}{|l|l|}
        \hline
        \textbf{Endpoint} & \texttt{GET /api/v1/jobs/\{job\_id\}} \\
        \hline
        \textbf{Authentication} & ต้อง (Bearer Token) \\
        \hline
    \end{tabular}
\end{table}

\textbf{Response - กำลังประมวลผล (200 OK):}
\begin{verbatim}
{
  "success": true,
  "data": {
    "job_id": 1,
    "image_id": 1,
    "status": "processing",
    "ai_model_version": "v1.2.0",
    "started_at": "datetime",
    "progress_percent": 45
  }
}
\end{verbatim}

\textbf{Response - เสร็จสมบูรณ์ (200 OK):}
\begin{verbatim}
{
  "success": true,
  "data": {
    "job_id": 1,
    "image_id": 1,
    "status": "completed",
    "ai_model_version": "v1.2.0",
    "started_at": "datetime",
    "finished_at": "datetime",
    "result_url": "/api/v1/jobs/1/result"
  }
}
\end{verbatim}

\textbf{Response - ล้มเหลว (200 OK):}
\begin{verbatim}
{
  "success": true,
  "data": {
    "job_id": 1,
    "image_id": 1,
    "status": "failed",
    "error_message": "Model inference timeout",
    "started_at": "datetime",
    "finished_at": "datetime"
  }
}
\end{verbatim}

\clearpage

\subsubsection{ดูผลการวิเคราะห์ (Get Analysis Result)}

\begin{table}[h!]
    \centering
    \begin{tabular}{|l|l|}
        \hline
        \textbf{Endpoint} & \texttt{GET /api/v1/jobs/\{job\_id\}/result} \\
        \hline
        \textbf{Authentication} & ต้อง (Bearer Token) \\
        \hline
    \end{tabular}
\end{table}

\textbf{Response (200 OK):}
\begin{verbatim}
{
  "success": true,
  "data": {
    "result_id": 1,
    "job_id": 1,
    "image_id": 1,
    "counts": {
      "viable": 150,
      "apoptosis": 23,
      "other": 8
    },
    "total_cells": 181,
    "avg_confidence_score": 0.92,
    "percentages": {
      "viable": 82.87,
      "apoptosis": 12.71,
      "other": 4.42
    },
    "raw_data": {
      "bounding_boxes": [
        {
          "class": "viable",
          "confidence": 0.95,
          "x": 100, "y": 200,
          "width": 50, "height": 50
        }
      ]
    },
    "summary_data": "พบเซลล์ทั้งหมด 181 เซลล์ ...",
    "analyzed_at": "datetime"
  }
}
\end{verbatim}

\clearpage

\subsubsection{ดูประวัติการวิเคราะห์ของรูปภาพ (Image Analysis History)}

\begin{table}[h!]
    \centering
    \begin{tabular}{|l|l|}
        \hline
        \textbf{Endpoint} & \texttt{GET /api/v1/images/\{image\_id\}/analysis-history} \\
        \hline
        \textbf{Authentication} & ต้อง (Bearer Token) \\
        \hline
    \end{tabular}
\end{table}

\textbf{Response (200 OK):}
\begin{verbatim}
{
  "success": true,
  "data": {
    "image_id": 1,
    "analyses": [
      {
        "job_id": 2,
        "status": "completed",
        "ai_model_version": "v1.3.0",
        "counts": {
          "viable": 155,
          "apoptosis": 20,
          "other": 6
        },
        "avg_confidence_score": 0.94,
        "finished_at": "2024-01-20T14:30:00Z"
      },
      {
        "job_id": 1,
        "status": "completed",
        "ai_model_version": "v1.2.0",
        "counts": {
          "viable": 150,
          "apoptosis": 23,
          "other": 8
        },
        "avg_confidence_score": 0.92,
        "finished_at": "2024-01-15T10:30:00Z"
      }
    ],
    "total": 2
  }
}
\end{verbatim}

\clearpage

%% ==================== API SUMMARY TABLE ====================
\subsection{สรุปตาราง API Endpoints}

\begin{table}[h!]
    \centering
    \small
    \begin{tabular}{|l|l|l|p{4cm}|}
        \hline
        \textbf{Method} & \textbf{Endpoint} & \textbf{Auth} & \textbf{Description} \\
        \hline
        \multicolumn{4}{|c|}{\textbf{Authentication}} \\
        \hline
        POST & /auth/register & No & ลงทะเบียนผู้ใช้ใหม่ \\
        POST & /auth/login & No & เข้าสู่ระบบ \\
        POST & /auth/logout & Yes & ออกจากระบบ \\
        \hline
        \multicolumn{4}{|c|}{\textbf{Folder Management}} \\
        \hline
        GET & /folders & Yes & รายการโฟลเดอร์ทั้งหมด \\
        POST & /folders & Yes & สร้างโฟลเดอร์ใหม่ \\
        PATCH & /folders/\{id\} & Yes & แก้ไขชื่อโฟลเดอร์ \\
        DELETE & /folders/\{id\} & Yes & ลบโฟลเดอร์ \\
        \hline
        \multicolumn{4}{|c|}{\textbf{Image Management}} \\
        \hline
        \hline
        GET & /folders/\{id\}/images & Yes & รายการภาพ (V1/V2) \\
        POST & .../request-upload & Yes & ขอ URL อัปโหลด (S3) \\
        POST & .../confirm-upload & Yes & ยืนยันการอัปโหลด \\
        GET & .../download-url & Yes & ขอ URL ดาวน์โหลด (S3) \\
        GET & /images/\{id\} & Yes & รายละเอียดภาพ \\
        DELETE & /images/\{id\} & Yes & ลบภาพ \\
        \hline
        \multicolumn{4}{|c|}{\textbf{AI Analysis}} \\
        \hline
        POST & /images/\{id\}/analyze & Yes & ส่งภาพเข้าวิเคราะห์ \\
        GET & /jobs/\{id\} & Yes & ตรวจสอบสถานะ Job \\
        GET & /jobs/\{id\}/result & Yes & ดูผลการวิเคราะห์ \\
        GET & /images/\{id\}/analysis-history & Yes & ประวัติการวิเคราะห์ \\
        \hline
    \end{tabular}
    \caption{สรุป API Endpoints ทั้งหมด (Base URL: /api/v1)}
    \label{tab:api_summary}
\end{table}
