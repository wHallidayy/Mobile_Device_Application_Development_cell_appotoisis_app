\section{ออกแบบฐานข้อมูล}

\subsection{แผนผังความสัมพันธ์เชิงนามธรรม (Physical Entity-Relationship Diagram)}

\begin{figure}[ht]
    \centering
    \includegraphics[width=0.95\textwidth]{assets/physical_erd.pdf} 
    \caption{Physical Entity-Relationship Diagram (Crow's Foot Notation) แสดงโครงสร้างตารางจริงในฐานข้อมูล PostgreSQL}
    \label{fig:physical_erd}
\end{figure}

\subsection{พจนานุกรมข้อมูล (Data Dictionary)}
ในส่วนนี้แสดงรายละเอียดของแต่ละแอตทริบิวต์ (Attribute) ในตารางฐานข้อมูล รวมถึงชนิดข้อมูล (Data Type) และความสัมพันธ์ (Key)

% --- Table: Users ---
\begin{table}[h!]
    \centering
    \textbf{ตารางข้อมูลผู้ใช้งาน (Users)}
    \label{tab:dd_users}
    \begin{tabular}{|p{3cm}|p{2.5cm}|p{1.5cm}|p{6cm}|}
        \hline
        \textbf{Attribute Name} & \textbf{Data Type} & \textbf{Key} & \textbf{Description} \\
        \hline
        user\_id & UUID & PK & รหัสระบุตัวตนผู้ใช้งาน (Primary Key) \\
        \hline
        username & VARCHAR & UQ & ชื่อบัญชีผู้ใช้งาน (ห้ามซ้ำ) \\
        \hline
        password\_hash & VARCHAR & & รหัสผ่านที่ผ่านการเข้ารหัส \\
        \hline
        role & ENUM & & บทบาทผู้ใช้ ('researcher', 'student', 'lecturer') \\
        \hline
        created\_at & DATETIME & & วันที่และเวลาที่สร้างบัญชี \\
        \hline
    \end{tabular}
\end{table}

% --- Table: Folders ---
\begin{table}[h!]
    \centering
    \textbf{ตารางข้อมูลโฟลเดอร์ (Folders)}
    \label{tab:dd_folders}
    \begin{tabular}{|p{3cm}|p{2.5cm}|p{1.5cm}|p{6cm}|}
        \hline
        \textbf{Attribute Name} & \textbf{Data Type} & \textbf{Key} & \textbf{Description} \\
        \hline
        folder\_id & INT & PK & รหัสโฟลเดอร์ (Auto Increment) \\
        \hline
        user\_id & UUID & FK & รหัสผู้ใช้งานที่เป็นเจ้าของ (อ้างอิงตาราง Users) \\
        \hline
        folder\_name & VARCHAR & & ชื่อโฟลเดอร์ \\
        \hline
        created\_at & DATETIME & & วันที่สร้างโฟลเดอร์ \\
        \hline
    \end{tabular}
\end{table}

% --- Table: Images ---
\begin{table}[h!]
    \centering
    \textbf{ตารางข้อมูลรูปภาพ (Images)}
    \label{tab:dd_images}
    \begin{tabular}{|p{3cm}|p{2.5cm}|p{1.5cm}|p{6cm}|}
        \hline
        \textbf{Attribute Name} & \textbf{Data Type} & \textbf{Key} & \textbf{Description} \\
        \hline
        image\_id & BIGINT & PK & รหัสรูปภาพ (Auto Increment) \\
        \hline
        folder\_id & INT & FK & รหัสโฟลเดอร์ที่เก็บภาพ (อ้างอิงตาราง Folders) \\
        \hline
        file\_path & VARCHAR & & ที่อยู่ไฟล์ภาพในระบบ Storage \\
        \hline
        original\_filename & VARCHAR & & ชื่อไฟล์เดิมที่อัปโหลด \\
        \hline
        mime\_type & VARCHAR & & ชนิดไฟล์ (เช่น image/jpeg) \\
        \hline
        file\_size & INT & & ขนาดไฟล์ (Bytes) \\
        \hline
        metadata & JSON/JSONB & & ข้อมูลจำเพาะของภาพ (ความละเอียด, วันที่ถ่าย) \\
        \hline
        uploaded\_at & DATETIME & & วันที่อัปโหลดภาพ \\
        \hline
    \end{tabular}
\end{table}

\textbf{Note:} ความสัมพันธ์ Foreign Key ทั้งหมดถูกกำหนดเป็น ON DELETE CASCADE 
เพื่อให้เมื่อลบรูปภาพ ข้อมูลงานและผลลัพธ์ที่เกี่ยวข้องจะถูกลบออกโดยอัตโนมัติ

\clearpage

% --- Table: Jobs ---
\begin{table}[h!]
    \centering
    \textbf{ตารางงานประมวลผล (Jobs)}
    \label{tab:dd_jobs}
    \begin{tabular}{|p{3cm}|p{2.5cm}|p{1.5cm}|p{6cm}|}
        \hline
        \textbf{Attribute Name} & \textbf{Data Type} & \textbf{Key} & \textbf{Description} \\
        \hline
        job\_id & BIGINT & PK & รหัสงานประมวลผล (Auto Increment) \\
        \hline
        image\_id & BIGINT & FK & รหัสรูปภาพที่ถูกประมวลผล (อ้างอิงตาราง Images) \\
        \hline
        status & ENUM & & สถานะงาน ('pending', 'processing', 'completed', 'failed') \\
        \hline
        ai\_model\_version & VARCHAR & & เวอร์ชันของโมเดล AI ที่ใช้ \\
        \hline
        started\_at & DATETIME & & เวลาเริ่มประมวลผล \\
        \hline
        finished\_at & DATETIME & & เวลาประมวลผลเสร็จสิ้น \\
        \hline
        error\_message & TEXT & & ข้อความแจ้งเตือนข้อผิดพลาด (ถ้ามี) \\
        \hline
    \end{tabular}
\end{table}

% --- Table: Analysis_Results ---
\begin{table}[h!]
    \centering
    \textbf{ตารางผลลัพธ์การวิเคราะห์ (Analysis\_Results)}
    \label{tab:dd_results}
    \begin{tabular}{|p{3.5cm}|p{2.2cm}|p{1.3cm}|p{6cm}|}
        \hline
        \textbf{Attribute Name} & \textbf{Data Type} & \textbf{Key} & \textbf{Description} \\
        \hline
        result\_id & BIGINT & PK & รหัสผลลัพธ์ (Auto Increment) \\
        \hline
        job\_id & BIGINT & FK, UQ & รหัสงานประมวลผล (อ้างอิงตาราง Jobs) \\
        \hline
        count\_viable & INT & & จำนวนเซลล์ปกติ (Default: 0) \\
        \hline
        count\_apoptosis & INT & & จำนวนเซลล์ที่กำลังตาย (Default: 0) \\
        \hline
        count\_other & INT & & จำนวนเซลล์อื่นๆ (Default: 0) \\
        \hline
        avg\_confidence\_score & FLOAT & & ค่าความเชื่อมั่นเฉลี่ย (0.0 - 1.0) \\
        \hline
        raw\_data & JSON/JSONB & & พิกัด Bounding Boxes ของวัตถุ \\
        \hline
        summary\_data & TEXT/JSON & & ข้อความสรุปผลสำหรับแสดงหน้าเว็บ \\
        \hline
    \end{tabular}
\end{table}

% --- Table: Database Indexes ---
\begin{table}[h!]
    \centering
    \textbf{ดัชนีฐานข้อมูล (Database Indexes)}
    \label{tab:dd_indexes}
    \begin{tabular}{|p{4cm}|p{4cm}|p{5cm}|}
        \hline
        \textbf{Index Name} & \textbf{Columns} & \textbf{Objective} \\
        \hline
        idx\_folders\_user\_id & (user\_id) & เพิ่มความเร็วในการตรวจสอบความเป็นเจ้าของ Folder \\
        \hline
        idx\_images\_folder\_listing & (folder\_id, uploaded\_at DESC) & เพิ่มความเร็วในการแสดงรายการรูปภาพแบบแบ่งหน้า (Pagination) \\
        \hline
        idx\_jobs\_image\_id & (image\_id) & รองรับการค้นหา Job จากรูปภาพ และเพิ่มความเร็ว Cascade Delete \\
        \hline
        idx\_jobs\_status & (status) & ใช้สำหรับดึงงานที่รอประมวลผล (Queue Polling) \\
        \hline
        idx\_results\_job\_id & (job\_id) & เพิ่มความเร็วในการดึงผลลัพธ์จาก Job ID \\
        \hline
    \end{tabular}
\end{table}

\subsection{กลยุทธ์การจัดเก็บข้อมูลและไฟล์ (Data \& File Storage Strategy)}
เนื่องจากระบบต้องจัดการกับไฟล์ภาพที่มีขนาดใหญ่ ผู้ออกแบบจึงแยกการจัดเก็บข้อมูลออกเป็น 2 ส่วนเพื่อประสิทธิภาพสูงสุด:
\begin{enumerate}
    \item \textbf{Structured Data \& Metadata:} ข้อมูลโครงสร้าง (Users, Jobs) และข้อมูล JSON (Metadata) จะถูกจัดเก็บในฐานข้อมูล PostgreSQL
    \item \textbf{Binary Data (Image Files):} ไฟล์รูปภาพจริงจะถูกจัดเก็บในระบบไฟล์ (File System) ของเซิร์ฟเวอร์ หรือ Object Storage เพื่อลดภาระ (Load) ของฐานข้อมูล โดยในตาราง \texttt{Images} จะจัดเก็บเพียง \texttt{file\_path} เพื่อชี้ตำแหน่งไฟล์เท่านั้น
\end{enumerate}

\subsection{ข้อกำหนดความคงสภาพของข้อมูล (Referential Integrity Constraints)}
เพื่อให้ข้อมูลมีความสอดคล้องกันทั้งระบบ และป้องกันปัญหาข้อมูลกำพร้า (Orphan Data) ได้กำหนดกฎการจัดการข้อมูลเมื่อมีการลบ (Delete Rule) แบบ \textbf{Cascade Delete} ในความสัมพันธ์หลัก ดังนี้:
\begin{itemize}
    \item \textbf{Users $\rightarrow$ Folders:} หากลบบัญชีผู้ใช้ โฟลเดอร์ทั้งหมดของผู้ใช้นั้นจะถูกลบ
    \item \textbf{Folders $\rightarrow$ Images:} หากลบโฟลเดอร์ ภาพทั้งหมดในโฟลเดอร์นั้นจะถูกลบ
    \item \textbf{Jobs $\rightarrow$ Analysis\_Results:} หากลบงานประมวลผล ผลลัพธ์การวิเคราะห์จะถูกลบตามไปด้วย
\end{itemize}

\subsection{กลยุทธ์การเพิ่มประสิทธิภาพการสืบค้น (Indexing Strategy)}
เพื่อให้ระบบตอบสนองได้รวดเร็วเมื่อปริมาณข้อมูลเพิ่มขึ้น ได้มีการกำหนดสร้างดัชนี (Index) ไว้ดังนี้:
\begin{itemize}
    \item \textbf{B-Tree Index:} สร้างบนคอลัมน์ Foreign Key ทั้งหมด (\texttt{user\_id}, \texttt{folder\_id}, \texttt{job\_id}) เพื่อความเร็วในการเชื่อมโยงตาราง (JOIN)
    \item \textbf{Unique Index:} สร้างบนคอลัมน์ \texttt{username} เพื่อตรวจสอบความซ้ำซ้อนและเพิ่มความเร็วในการเข้าสู่ระบบ (Login)
\end{itemize}

\clearpage

\section{แผนการย้ายข้อมูลและการปรับปรุงฐานข้อมูล (Database Migration Plan)}

ส่วนนี้อธิบายถึงแผนการนำสคริปต์ SQL ไปใช้งานจริง (Deployment) และการปรับแต่งดัชนี (Index Optimization) เพื่อให้ระบบสามารถรองรับข้อมูลขนาดใหญ่ได้ตามการวิเคราะห์ Big O Notation

\subsection{การปรับจูนประสิทธิภาพ (Performance Tuning)}

จากการวิเคราะห์ประสิทธิภาพการทำงาน (Performance Analysis) ได้มีการปรับปรุงโครงสร้าง Index จากการออกแบบตั้งต้นดังนี้:

\subsubsection{1. การปรับปรุง Index สำหรับ Pagination}
\begin{itemize}
    \item \textbf{เดิม:} \texttt{CREATE INDEX idx\_images\_folder\_id ON images(folder\_id);}
    \item \textbf{ปัญหา:} เมื่อผู้ใช้งานเปิดดูรูปภาพในโฟลเดอร์ ระบบจะต้องทำการ Sort ข้อมูลตามเวลา (\texttt{uploaded\_at}) ใหม่ทุกครั้ง ซึ่งมี Time Complexity เป็น $O(N \log N)$
    \item \textbf{ปรับปรุง:} เปลี่ยนเป็น \textbf{Composite Index} \texttt{(folder\_id, uploaded\_at DESC)}
    \item \textbf{ผลลัพธ์:} การดึงข้อมูลรูปภาพหน้าละ 20 รูป จะทำได้รวดเร็วและคงที่ $O(K + \log N)$ เสมอ ไม่ว่าจะมีรูปภาพในโฟลเดอร์จำนวนมากเพียงใด
\end{itemize}

\subsubsection{2. การปรับปรุง Index สำหรับ AI Worker Queue}
\begin{itemize}
    \item \textbf{เดิม:} \texttt{CREATE INDEX idx\_jobs\_status ON jobs(status);}
    \item \textbf{ปัญหา:} AI Worker ประมวลผลงานแบบ FIFO (First In, First Out) ซึ่งต้องค้นหางานสถานะ 'pending' ที่เก่าที่สุด Database อาจต้อง Scan เพื่อหาลำดับที่ถูกต้อง
    \item \textbf{ปรับปรุง:} เพิ่ม \texttt{created\_at ASC} เข้าไปใน Index เป็น \texttt{(status, created\_at ASC)}
    \item \textbf{ผลลัพธ์:} คำสั่งดึงงาน \texttt{ORDER BY created\_at ASC LIMIT 1} จะทำงานได้เร็วที่สุด
\end{itemize}

\subsection{สคริปต์ฐานข้อมูลฉบับสมบูรณ์ (Optimized Database Script)}

สคริปต์ SQL ต่อไปนี้รวมการปรับปรุงทั้งหมดและพร้อมสำหรับการใช้งานจริง:

\begin{lstlisting}[language=SQL, caption=Optimized SQL Schema with High Performance Indexes]
-- เปิดใช้งาน UUID Extension
CREATE EXTENSION IF NOT EXISTS "uuid-ossp";

-- สร้าง Enum Types
CREATE TYPE user_role AS ENUM ('researcher', 'student', 'lecturer');
CREATE TYPE job_status AS ENUM ('pending', 'processing', 'completed', 'failed');

-- 1. ตาราง Users
CREATE TABLE users (
    user_id UUID PRIMARY KEY DEFAULT uuid_generate_v4(),
    username VARCHAR(255) UNIQUE NOT NULL,
    password_hash VARCHAR(255) NOT NULL,
    role user_role NOT NULL DEFAULT 'student',
    created_at TIMESTAMPTZ DEFAULT CURRENT_TIMESTAMP
);

-- 2. ตาราง Folders
CREATE TABLE folders (
    folder_id SERIAL PRIMARY KEY,
    user_id UUID NOT NULL REFERENCES users(user_id) ON DELETE CASCADE,
    folder_name VARCHAR(255) NOT NULL,
    created_at TIMESTAMPTZ DEFAULT CURRENT_TIMESTAMP,
    UNIQUE (user_id, folder_name) -- ป้องกันชื่อโฟลเดอร์ซ้ำใน user เดียวกัน
);

-- 3. ตาราง Images
CREATE TABLE images (
    image_id BIGSERIAL PRIMARY KEY,
    folder_id INT NOT NULL REFERENCES folders(folder_id) ON DELETE CASCADE,
    file_path VARCHAR(500) NOT NULL,
    original_filename VARCHAR(255) NOT NULL,
    mime_type VARCHAR(50) NOT NULL,
    file_size INT NOT NULL,
    metadata JSONB DEFAULT '{}'::jsonb,
    uploaded_at TIMESTAMPTZ DEFAULT CURRENT_TIMESTAMP
);

-- 4. ตาราง Jobs
CREATE TABLE jobs (
    job_id BIGSERIAL PRIMARY KEY,
    image_id BIGINT NOT NULL REFERENCES images(image_id) ON DELETE CASCADE,
    status job_status NOT NULL DEFAULT 'pending',
    ai_model_version VARCHAR(50),
    started_at TIMESTAMPTZ,
    finished_at TIMESTAMPTZ,
    error_message TEXT,
    created_at TIMESTAMPTZ DEFAULT CURRENT_TIMESTAMP
);

-- 5. ตาราง Analysis_Results
CREATE TABLE analysis_results (
    result_id BIGSERIAL PRIMARY KEY,
    job_id BIGINT UNIQUE NOT NULL REFERENCES jobs(job_id) ON DELETE CASCADE,
    count_viable INT DEFAULT 0,
    count_apoptosis INT DEFAULT 0,
    count_other INT DEFAULT 0,
    avg_confidence_score FLOAT,
    raw_data JSONB, -- เก็บ Bounding Boxes
    summary_data TEXT,
    analyzed_at TIMESTAMPTZ DEFAULT CURRENT_TIMESTAMP
);

-- ==========================================
-- OPTIMIZED INDEXES (High Performance)
-- ==========================================

-- 1. ตรวจสอบความเป็นเจ้าของ Folder เร็วๆ
CREATE INDEX idx_folders_user_id ON folders(user_id);

-- 2. Pagination: ดึงรูปใน Folder เรียงตามเวลา (สำคัญมาก)
CREATE INDEX idx_images_folder_listing ON images(folder_id, uploaded_at DESC);

-- 3. Integrity & Cascade Delete Support
CREATE INDEX idx_jobs_image_id ON jobs(image_id);

-- 4. Queue Processing: ดึงงานเก่าสุดที่รออยู่ (FIFO)
CREATE INDEX idx_jobs_queue ON jobs(status, created_at ASC);

-- 5. Result Lookup: ดึงผลลัพธ์จาก Job ID
CREATE INDEX idx_results_job_id ON analysis_results(job_id);
\end{lstlisting}