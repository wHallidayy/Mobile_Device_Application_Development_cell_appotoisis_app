\section{การดำเนินการและติดตั้งระบบ (Implementation \& Deployment)}

ในบทนี้จะกล่าวถึงเครื่องมือและขั้นตอนในการติดตั้งสภาพแวดล้อมสำหรับการพัฒนา (Development Environment) และการนำระบบขึ้นใช้งานจริง (Deployment) โดยเน้นการใช้งาน Docker Compose เพื่อจัดการ Container ของบริการต่างๆ และ Makefile สำหรับการทำงานอัตโนมัติ

\subsection{เครื่องมือที่ใช้ (Prerequisites)}

ก่อนเริ่มต้นการพัฒนาหรือติดตั้งระบบ จำเป็นต้องมีเครื่องมือดังต่อไปนี้:

\begin{itemize}
    \item \textbf{Docker Desktop} หรือ \textbf{Docker Engine}: สำหรับรัน Container
    \item \textbf{Rust Toolchain}: สำหรับ Build และ Run Backend Application
    \item \textbf{Make}: สำหรับรัน Command Automation
    \item \textbf{SQLx CLI}: สำหรับจัดการ Database Migrations (ติดตั้งด้วยคำสั่ง \texttt{cargo install sqlx-cli})
\end{itemize}

\subsection{การจำลองสภาพแวดล้อมด้วย Docker (Containerization)}

ระบบได้ออกแบบให้รันบน Docker เพื่อความสะดวกในการติดตั้งและรับประกันว่าสภาพแวดล้อมจะเหมือนกันในทุกเครื่อง โดยใช้ \texttt{docker-compose.yml} ในการกำหนด Service ต่างๆ ดังนี้:

\begin{enumerate}
    \item \textbf{postgres}: ฐานข้อมูล PostgreSQL เวอร์ชัน 16-alpine
    \item \textbf{minio}: Object Storage สำหรับเก็บรูปภาพ (S3 Compatible)
    \item \textbf{minio-setup}: Container ชั่วคราวสำหรับสร้าง Bucket เริ่มต้นใน MinIO
    \item \textbf{pgadmin}: Web Interface สำหรับจัดการฐานข้อมูล PostgreSQL
\end{enumerate}

\clearpage

\begin{listing}
    \caption{ไฟล์ docker-compose.yml}
    \label{lst:docker-compose}
\begin{verbatim}
version: "3.8"

services:
  postgres:
    image: postgres:16-alpine
    container_name: ${PROJECT_NAME:-myproject}-postgres
    ports:
      - "${POSTGRES_PORT:-5432}:5432"
    environment:
      POSTGRES_USER: ${POSTGRES_USER:-postgres}
      POSTGRES_PASSWORD: ${POSTGRES_PASSWORD:-postgres}
      POSTGRES_DB: ${POSTGRES_DB:-mydb}
    volumes:
      - postgres-data:/var/lib/postgresql/data

  minio:
    image: minio/minio:latest
    container_name: ${PROJECT_NAME:-myproject}-minio
    ports:
      - "${MINIO_PORT:-9000}:9000"
      - "${MINIO_CONSOLE_PORT:-9001}:9001"
    environment:
      MINIO_ROOT_USER: ${MINIO_ROOT_USER:-minioadmin}
      MINIO_ROOT_PASSWORD: ${MINIO_ROOT_PASSWORD:-minioadmin}
    command: server /data --console-address ":9001"

\end{verbatim}
\end{listing}

\subsection{ระบบอัตโนมัติด้วย Makefile (Automation)}

เพื่อความสะดวกในการสั่งงานระบบ ได้มีการสร้าง \texttt{Makefile} รวบรวมคำสั่งที่ใช้บ่อยไว้ แบ่งเป็นหมวดหมู่ดังนี้:

\subsubsection{คำสั่งจัดการ Docker}
\begin{ul}
    \item \texttt{make up}: สั่ง Start Containers ทั้งหมด
    \item \texttt{make down}: สั่ง Stop และ Remove Containers
    \item \texttt{make down-v}: สั่ง Stop และลบ Volumes (ลบข้อมูลทั้งหมด)
    \item \texttt{make logs}: ดู Logs ของทุก Service
\end{ul}

\subsubsection{คำสั่งสำหรับ Backend (Rust)}
\begin{ul}
    \item \texttt{make rust-run}: รัน Backend Server (Development Mode)
    \item \texttt{make rust-build}: Build Project
    \item \texttt{make rust-test}: รัน Unit Tests และ Integration Tests
    \item \texttt{make rust-audit}: ตรวจสอบช่องโหว่ความปลอดภัยของ Dependencies
\end{ul}

\clearpage

\begin{listing}
    \caption{ตัวอย่าง Makefile}
    \label{lst:makefile}
\begin{verbatim}
# Start all containers
up:
    docker compose up -d

# Stop all containers
down:
    docker compose down

# Run backend dev server
rust-run:
    cd backend && cargo run

# Database Migrations
sqlx-run:
    cd backend && sqlx migrate run
\end{verbatim}
\end{listing}

\subsection{การจัดการฐานข้อมูล (Database Migrations)}

ระบบใช้ \textbf{SQLx Migrations} ในการจัดการโครงสร้างฐานข้อมูล (Schema Evolution) ทำให้สามารถติดตามการเปลี่ยนแปลงของ Database schema ได้ในรูปแบบ Version Control

\subsubsection{ขั้นตอนการทำ Migration}
\begin{enumerate}
    \item \textbf{สร้างไฟล์ Migration ใหม่}:
    \begin{verbatim}
    make sqlx-create msg='create_users_table'
    \end{verbatim}
    คำสั่งนี้จะสร้างไฟล์ \texttt{.sql} ในโฟลเดอร์ \texttt{migrations/} โดยมี Timestamp นำหน้า

    \item \textbf{เขียนคำสั่ง SQL}:
    ในไฟล์ที่สร้างขึ้น ให้ใส่คำสั่ง SQL สำหรับสร้างตาราง (Up) และลบตาราง (Down) ตัวอย่าง:
    \input{components/sql-create-table.tex}

    \item \textbf{อัปเดตฐานข้อมูล (Apply Migrations)}:
    \begin{verbatim}
    make sqlx-run
    \end{verbatim}

    \item \textbf{ย้อนกลับ (Revert)}:
    หากมีความผิดพลาด สามารถย้อนกลับ Migration ล่าสุดได้:
    \begin{verbatim}
    make sqlx-revert
    \end{verbatim}
\end{enumerate}

\subsection{การตั้งค่า Environment Variables}

ระบบใช้ไฟล์ \texttt{.env} ในการเก็บค่า configuration ต่างๆ โดยมีไฟล์ \texttt{.env.example} เป็นต้นแบบ ผู้ใช้งานต้องสร้างไฟล์ \texttt{.env} และกำหนดค่าดังนี้:

\begin{verbatim}
# Database
POSTGRES_USER=postgres
POSTGRES_PASSWORD=password
POSTGRES_DB=app_db

# MinIO
MINIO_ROOT_USER=minioadmin
MINIO_ROOT_PASSWORD=minioadmin
\end{verbatim}